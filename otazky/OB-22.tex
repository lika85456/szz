\subsection{OB-22 (UKB)}
Bezpečnost kyber-fyzikálních systémů a Internetu věcí, specifické hrozby a specifika zabezpečení (ve srovnání s tradiční IT bezpečnosti). Možnosti detekce útoků vedených proti kyber-fyzikálních systémů. Purdue model pro informační a komunikační systémy (ICS) a specifika jejich bezpečnosti.

\begin{itemize}
	\item Internet of Things --- síť chytrých zařízení (chytrá domácnost, chytrá města, kamery, senzory, chytrá auta, ...)
	\item CPS (Cyber-Physical Systems) --- propojuje fyzická zařízení s IT, může ovládat/monitorovat fyzické procesy v reálném čase (průmyslová zařízení, zdravotní vybavení, dopravní systémy...)
	\item rozdíly Iot, CPS oproti tradiční IT bezpečnosti:
	\begin{itemize}
		\item selhání může způsobit fyzické škody/smrt
		\item cíle útočníků typicky bývají způsobit škody
		\item IoT zařízení mají většinou omezené zdroje (pomalé CPU, malou paměť...), tedy nelze vždy aplikovat stejná bezpečnostní řešení
		\item zařízení mohou být provozována lidmi bez znalosti technologie/bezpečnosti
		\item výrobní náklady jsou tlačeny na minimum
		\item doba na návrh, vývoj, implementaci, vývoj a vyřazení z provozu jsou v řádu desetiletí (např. elek\-trár\-ny)
	\end{itemize}
	\item možnosti prevence útoků:
	\begin{itemize}
		\item lze i tradičním přístupem (firewall atd), ale je to nepraktické pro IoT
		\item navrhovat systémy, ve kterých lze zabezpečení průběžně aktualizovat
		\item zajistit dodatečná bezpečnostní řešení pro stávající starší systémy
		\item přidání extra zařízení "na drát", které kontroluje pakety
		\item extra zařízení blokující bezdrátové připojení neautorizovaných zařízení
		\item "lehké" (lightweight) kryptografické algoritmy 
		\item bezpečná mikrojádra (bezpečný malý OS)
	\end{itemize}
	\item detekce útoků na CPS
	\begin{itemize}
		\item důvěryhodný ověřovatel vnitřního stavu IoT zařízení, např. kontrola obsahu RAM
		\item sledování interakcí CPS (něco jako IDS v klasických sítích, ale jednodušší, protože komunikace je méně pestrá a stabilnější)
		\item sledování fyzických stavů / konfigurací efektorů --- např konfigurace co jsme zatím neviděli, nebo fyzikálně nesmyslná data senzoru
		\item aktivní detekce --- pravidelné dotazování zařízení, detekce anomálií v odpovědích
	\end{itemize}
	\item mitigace
	\begin{itemize}
		\item konzervativní kontrola --- provozování systému s dostatečnými bezpečnostními rezervami
		\item kontrola senzorů navzájem --- zda data dávají smysl
		\item omezení aktivace --- v případě útočníka v systému chvíli trvá než může nějak zasáhnout a něco změnit
		\item kontrola akcí systému --- referenční monitor, nepovede akce k nebezpečnému chování?
	\end{itemize}
\end{itemize}