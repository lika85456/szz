\subsection{OB-25 (ZSB)}
Řízení přístupu v operačních systémech, obecný model řízení přístupu - Trusted Computing Base, vícestupňové (multilevel) a multilaterální modely, přístupy Discretionary Access Control a Mandatory Access Control, kon\-krét\-ní příklady implementace v OS.

\subsubsection*{Model hrozeb OS}
\begin{itemize}
	\item důvernost --- ztráta dat procesu nebo uživatelské přihlašovací údaje
	\item integrita --- úprava přístupových práv, úprava dat jiného procesu, instalace malware
	\item dostupnost --- pád systému, využití všech systémových zdrojů, způsobení nedostupnosti dat
	\item možné útoky:
	\begin{itemize}
		\item škodlivý software --- boot kits, škodlivá rozšíření (drivery, moduly)
		\item chyby a bugy --- paměťové poškození, paměťové chyby, leaknutí dat
		\item postranní kanály --- hardwarové, spekulativní, softwarové
		\item DoS --- vyčerpání zdrojů, deadlock
	\end{itemize}
	\item vektory útoku:
	\begin{itemize}
		\item kód běžící v user space
		\item rozšíření OS
		\item kód z internetu (např. javascript)
		\item škodlivé periferie
		\item vzdálený systém
	\end{itemize}
	\item různé (historicky) návrhy OS
	\begin{itemize}
		\item single domain --- žádná izolace, vše běží přímo na HW
		\item monolitické OS --- moderní univerzílní OS, každá aplikace ve vlastní bezpečnostní doméně, OS jako celek je pod všemi aplikacemi
		\item microkernel based multi-server OS --- všechny součásti OS mají vlastní doménu
		\item unikernel/library OS --- aplikace spolu s OS knihovnou ve svých doménách, pod nimi samotný plánovač
	\end{itemize}
\end{itemize}

TODO