\subsection{OB-25 (ZSB)}
Řízení přístupu v operačních systémech, obecný model řízení přístupu - Trusted Computing Base, vícestupňové (multilevel) a multilaterální modely, přístupy Discretionary Access Control a Mandatory Access Control, kon\-krét\-ní příklady implementace v OS.

\subsubsection*{Model hrozeb OS}
\begin{itemize}
	\item důvernost --- ztráta dat procesu nebo uživatelské přihlašovací údaje
	\item integrita --- úprava přístupových práv, úprava dat jiného procesu, instalace malware
	\item dostupnost --- pád systému, využití všech systémových zdrojů, způsobení nedostupnosti dat
	\item možné útoky:
	\begin{itemize}
		\item škodlivý software --- boot kits, škodlivá rozšíření (drivery, moduly)
		\item chyby a bugy --- paměťové poškození, paměťové chyby, leaknutí dat
		\item postranní kanály --- hardwarové, spekulativní, softwarové
		\item DoS --- vyčerpání zdrojů, deadlock
	\end{itemize}
	\item vektory útoku:
	\begin{itemize}
		\item kód běžící v user space
		\item rozšíření OS
		\item kód z internetu (např. javascript)
		\item škodlivé periferie
		\item vzdálený systém
	\end{itemize}
	\item různé (historicky) návrhy OS
	\begin{itemize}
		\item single domain --- žádná izolace, vše běží přímo na HW
		\item monolitické OS --- moderní univerzílní OS, každá aplikace ve vlastní bezpečnostní doméně, OS jako celek je pod všemi aplikacemi
		\item microkernel based multi-server OS --- všechny součásti OS mají vlastní doménu
		\item unikernel/library OS --- aplikace spolu s OS knihovnou ve svých doménách, pod nimi samotný plánovač
	\end{itemize}
\end{itemize}

\subsubsection*{Řízení přístupu v OS}
\begin{itemize}
	\item principy:
	\begin{itemize}
		\item izolace --- oddělení bezpečnostních domén
		\item domény by neměly mít mnoho společných mechanismů
		\item zprostředkování komunikace mezi doménami by mělo být fail-secure
		\item pouze povolené domény by měly mít přístup ke zdroji
	\end{itemize}
	\item primitiva pro izolaci a zprostředkování --- autentizace, autorizace, auditování, accountability (úč\-to\-va\-tel\-nost?)
	\item příklady izolace
	\begin{itemize}
		\item android --- každá aplikace má svoje user id, nemohou spolu komunikovat a mají omezený přístup k OS
		\item iOS --- sice stejné user-id, ale všechny aplikace běží v chroot jailu --- každá má svoje virtuální prostředí a nemůže přistupovat k souborům mimo
	\end{itemize}
	\item obecný model řízení přístupu --- Trusted Computing Base (TCB)
	\begin{itemize}
		\item TCB komponenty jsou zodpovědné za vynucování bezpečnostních politik
		\item Reference monitor --- komponenta TCB validující přístup ke zdrojům, vynucuje bezpečnostní politiky
		\item TCB musí být kompletní (vše musí být vyalidováno přes TCB), izolované (chráněné před nedovolenou změnou) a ověřitelné (TCB splňuje návrhové specifikace)
	\end{itemize}
	\item multilevel security model --- různé úrovně důvěrnosti informací (open , confidential, secret, top secret) --- pochází z armády
	\item multi-lateral security model --- odděluje různé kategorie informací
	\item bezpečnostní politiky --- kdo nastavuje?
	\begin{itemize}
		\item Discretionary Access Control (DAC) --- nastavuje vlastník  --- typický příklad: ACL
		\item Mandatory Access Control (MAC) --- v celém systému stejné politiky --- většinou víceúrovňový model
	\end{itemize}
	\item v SELinux le implementováno jak DAC tak MAC
	\item SELinux --- více kategorií, v každé různé úrovně důvěrnosti
	\item existuje i např. ve Windows, kontrola úrovně integrity (integrity level --- od untrusted (0) po installer (5))
\end{itemize}