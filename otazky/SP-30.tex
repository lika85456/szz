\subsection{SP-30 (SAP)}
Kódy pro zobrazení čísel se znaménkem a realizace aritmetických operací (paralelní sčítačka/odčítačka, realizace aritmetických posuvů, dekodér, multiplexor, čítač). Reprezentace čísel v pohyblivé řádové čárce.

\subsubsection*{Čísla se znaménkem}
Existuje několik možností, jak v počítači ukládat celá čísla:
\begin{itemize}
	\item Přímý kód
	\begin{itemize}
		\item první bit je znaménkový --- určuje tedy, zda je hodnota za ním kladná či záporná
		\item ostatní bity představují absolutní hodnotu čísla
		\item existuj zde kladná i záporná nula
	\end{itemize}
	
	\item Doplňkový kód
	\begin{itemize}
		\item dle prvního bitu lze poznat znaménko čísla
		\item převod kladné $\leftrightarrow$ záporné lze vysvětlit jako inverze bitů a následné přičtení jedničky
		\item 0111 (7) $\rightarrow$ 1001 (-7)
		\item není zde záporná nula
	\end{itemize}
	\item Aditivní kód
	\begin{itemize}
		\item uložené číslo je posunuto o nějakou konstantu, typicky polovina rozsahu
		\item pro 4 bity určeme nulu jako 1000 --- pak 1111 je 7, 0000 je -8
		\item nula není zobrazena jako nula
		\item není zde záporná nula
	\end{itemize}
\end{itemize}

\subsubsection*{Čísla v pohyblivé řádové čárce}
Reprezentace pohyblivé řádové čárky vychází ze zobrazení $A=M*z^e$ používaném např. ve fyzice, kde $z$ je základ soustavy (zde 2), $e$ je exponent jako celé číslo, $M$ je mantisa.
\begin{itemize}
	\item používá se normalizovaný tvar, tedy mantisa je zapsána tak, že ji nelze "posunout" více doleva.
	\item v přímém kódu mantisy je vlevo vždy jednička, která se skrývá (zvýšení přesnosti)
	\item pro mantisu se typicky používá přímý kód, pro exponent aditivní
	\item float (32b) typicky vypadá jako 1b znaménko, pak 8b exponent a nakonec 23b mantisa (tedy přesnost 24b)
\end{itemize}

\subsubsection*{Realizace aritmetických operací}
\begin{itemize}
	\item Paralelní sčítačka
	
	Tvořena více jednobitovými sčítačkami. Jednobitová sčítačka má 3 vstupy: A, B (sčítané bity) a vstupní přenos (carry --- např. ze sčítačky nižšího řádu). Výstupy jsou S (výsledek) a výstupní přenos.
	
	\item Aritmetické posuvy
	
	Posun čísla vlevo/vpravo.
	Realizuje posuvný registr. Existuje více různých posuvů:
	\begin{itemize}
		\item logický posuv --- doplňuje nuly
		\item cyklický posuv --- doplňuje co vylezlo na druhé straně
		\item aritmetický posuv --- doplňuje 1 nebo 0 podle znaménka čísla
	\end{itemize}
	
	\item Dekodér
	
	Kombinační logický obvod, který má méně bitů na vstupu než na výstupu, a podle tabulky převádí. Kodér má opačnou funkci.
	
	\item Multiplexor
	
	Na základě řídícího signálu vybere, který ze vstupů pošle na výstup (má několik vstupů + řídící vstup, a jeden výstup).
	
	\item Čítač
	
	Registr s funkcí inkrementu/dekrementu, může čítat nahoru a/nebo dolů. Existují úplné čítače (do mocnin 2) či neúplné (do jiných čísel). Typicky čítají v binárním kódu, lze i např. v Grayově kódu.
\end{itemize}