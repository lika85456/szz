\subsection{SP-19 (PA1)}
Datové typy v programovacích jazycích. Staticky a dynamicky alokované proměnné, spojové seznamy. Modulární programování, procedury a funkce, vstupní a výstupní parametry. Překladač, linker, debugger.

\subsubsection*{Datové typy}
V programovacích jazycích používáme proměnné, tedy něco, co uchovává datovou hodnotu s nějakou vnitřní strukturou. Proměnné jsou identifikovány svými jmény --- identifikátory.
Datový typ proměnné definuje vnitřní strukturu/reprezentaci dat a jejich význam. Tím určuje jakých hodnot může proměnná nabývat a také jaké operace lze s proměnnou (její hodnotou) vykonávat.

\textbf{Jednoduché datové typy:}
\begin{itemize}

	\item celočíselné
	
	\begin{itemize}
		\item existují různé délky --- short, int, long (byte, long long, ...)
		\item signed (znaménkové) --- umí uložit i záporné hodnoty, používá se doplňkový kód
		\item unsigned (neznaménkové) --- ukládá jen kladné hodnoty, přímý kód
	\end{itemize}
	
	\item s pohyblivou řádovou čárkou
	
	\begin{itemize}
		\item existují různé délky --- float, double, long double
		\item znaménko (1 bit) + mantisa (velikost=přesnost) + exponent (velikost=rozsah)
	\end{itemize}
	
	\item znakové
	
		Znaky jsou kódovány jako čísla, používá se ASCII / extended ASCII / UNICODE.
		
	\item logická hodnota
	
		Není v C, ale často se v jazycích vyskytuje (boolean --- true/false).
		
\end{itemize}

\textbf{Další datové typy:}
\begin{itemize}
	\item ukazatel (pointer)
	
		Adresy paměti, kde je uložen datový typ pointeru (pointer vždy ukazuje na konkrétní typ/funkci, případně void).
		
	\item výčtový typ (enum)
	
	\item struktura
	
		Je složena z dalších datových typů, klidně dalších struktur.
		
	\item union
	
		Ukládá více různých datových typů na stejné místo.
		
	\item třída (ve vyšších jazycích)
	
\end{itemize}

\subsubsection*{Statická a dynamická alokace}
Staticky alokované proměnné:
\begin{itemize}
	\item vzniknou běžnou deklarací
	\item ukládají se na zásobník (lokální proměnné) či do části .BSS (neinicializované globální proměnné) a .DATA (inicializované globální proměnné)
	\item v případě pole je nutno znát v době kompilace velikost (statická velikost)
\end{itemize}

Dynamicky alokované proměnné
\begin{itemize}
	\item vzniknou použitím speciální funkcí/operátorem
	\item ukládají se na haldě (heap)
	\item přistupujeme přes pointer
	\item je možné alokovat paměť podle hodnot spočítaných za běhu programu
\end{itemize}

\subsubsection*{Spojové seznamy}
\begin{itemize}
	\item oproti poli nejsou položky seřazeny v paměti, ale každý prvek seznamu obsahuje ukazatel na další prvek.
	\item podobně jako v dynamicky alokovaném poli lze ukládat předem neznámý objem dat
	\item nelze jednoduše indexovat, ale lze libovolně přidávat či ubírat prvky z jakékoliv pozice v seznamu
\end{itemize}

\subsubsection*{Modulární programování}
\begin{itemize}
	\item složitější programy mohou být rozděleny do modulů
	\item tyto moduly lze použít v různých dalších částech programu
	\item modul má svou specifikační část (deklarace poskytovaných prostředků/rozhraní) a implementační část (definice/implementace poskytovaných prostředků)
	\item v C/C++ typicky hlavičkový soubor (.h/.hpp) a implementační soubor (.c/.cpp)
\end{itemize}

\subsubsection*{Procedury, funkce a parametry}
\begin{itemize}
	\item procedura/funkce je posloupnost příkazů uložených v paměti programu
	\begin{itemize}
		\item procedura --- bez návratové hodnoty (typ void)
		\item funkce --- s návratovou hodnotou
	\end{itemize}
	\item použijeme ji zavoláním přes její jméno 
	\item deklarace je specifikace jejího rozhraní --- parametrů a typu návratové hodnoty
	\item definice je samotný kód funkce
	\item vstupní paramtery jsou informace, které využije kód funkce
	\item výstupní parametry jsou výsledkem běhu funkce --- typicky se nějak změní a tím nám dají výsledek
\end{itemize}

\subsubsection*{Překladač}
\begin{itemize}
	\item překládá vyšší programovací jazyky do nižších
	\item ze zdrojového kódu vzniká objektový soubor --- modul se strojovým kódem
	\item front-end přeloží konkrétní jazyk do vnitřní reprezentace (abstrakce nezávislá ani na platformě ani na jazyku)
	\item back-end přeloží vnitřní reprezentaci do strojového kódu konkrétní platformy
\end{itemize}

\subsubsection*{Linker}
\begin{itemize}
	\item spojuje přeložené moduly do výsledného celku --- programu
	\item výstupem je spustitelný soubor
\end{itemize}

\subsubsection*{Debugger}
\begin{itemize}
	\item usnadňuje hledání chyb v kódu, také usnadňuje pochopení programu
	\item je vhodné kompilovat s informacemi pro ladění
	\item je možné si na nějakém místě běh programu zastavit a např. sledovat obsah proměnných, pouštět každý krok programu postupně...
\end{itemize}
