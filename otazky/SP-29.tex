\subsection{SP-29 (SAP)}
Architektura číslicového počítače, instrukční cyklus počítače, základní třídy souborů instrukcí (ISA). Paměťový subsystém počítače, paměťová hierarchie, skrytá paměť (cache).

\subsubsection*{Architektura číslicového počítače}

\begin{itemize}
	\item architektura se zabývá strukturou a chováním počítače
	
	\item řeší specifikaci různých funkčních modulů jako procesor a paměť a nebo např. instrukční sadu
	
	\item podsměry:
	\begin{itemize}
		\item ISA --- Instruction Set Architecture --- architektura souboru instrukcí
		\item Mikroarchitektura --- konkrétní sestavení a složení procesoru
		\item Systémový design --- řeší další HW komponenty
	\end{itemize}
	
	\item každý počítač je složen z následujících částí:
	\begin{itemize}
		\item datová část procesoru --- ALU, registry
		\item řadič --- řídící jednotka procesoru
		\item paměťový subsystém
		\item vstupní zařízení
		\item výstupní zařízení
	\end{itemize}
	
	\item typy architektury:
	\begin{itemize}
		\item Von Neumannova architektura
		
		Data i instrukce jsou uložena spolu, nejsou explicitně označena/ny.
		
		\item Harvardská architektura
		
		Data a instrukce jsou rozdělené.
	\end{itemize}
\end{itemize}

\subsubsection*{Instrukční cyklus počítače}
\begin{itemize}
	\item čtení instrukce (IF --- Instruction Fetch)
	\item dekódování instrukce (ID --- Instruction Decode)
	\item načtení operandů (OF --- Operand Fetch)
	\item provedení instrukce (IE --- Instruction Execution)
	\item zapsání/uložení výsledku (WB --- Write Back / Result Store)
	\item přerušení?
\end{itemize}

\textbf{Co je instrukce?}
Obsahuje informace:
\begin{itemize}
	\item co se má provést
	\item s čím se to má provést (operandy)
	\item kam se má uložit výsledek
	\item kde se má pokračovat
\end{itemize}

Tyto informace mohou být zadány explicitně, nebo mohou být dány typem instrukce, tedy architekturou počítače --- tedy implicitně.

\paragraph*{ISA --- Architektura souboru instrukcí}

Co je potřeba určit:
\begin{itemize}
	\item typy a formáty instrukcí, instrukční soubor
	\item datové typy, kódování a reprezentace, způsob uložení dat v paměti
	\item módy adresování paměti a přístup do paměti dat a instrukcí
	\item mimořádné stavy
\end{itemize}
Výhody:
\begin{itemize}
	\item abstrakce --- možnost různě implementovat stejnou architekturu instrukcí
	\item definice rozhraní mezi nízkoúrovňovým AW a HW
	\item standardizuje instrukce, bitové vzory strojového jazyka
\end{itemize}

\subsubsection*{Třídy souborů instrukcí (ISA)}
\begin{itemize}
	\item Střadačově (akumulátorově) orientovaná ISA

	Akumulátor je registr pro mezivýpočty, používá se implicitně jako zdroj pro výpočty i jako cíl pro výsledky. Používají se instrukce s jedním operandem. Nejstarší ISA (1949-60) --- vyvinula se z kalkulaček.
	
	\textbf{Výhody:}
	\begin{itemize}
		\item jednoduchý HW
		\item minimální vnitřní stav procesoru --- rychlé přepínání kontextu
		\item krátké instrukce
		\item jednoduché dekódování instrukcí
	\end{itemize}
	
	\textbf{Nevýhody:}
	\begin{itemize}
		\item častá komunikace s pamětí
		\item omezený paralelismus mezi instrukcemi
	\end{itemize}	
	
	Populární v 50. --- 70. letech, HW byl drahý, paměť byla rychlejší než CPU.
	
	\item Zásobníkově orientovaná ISA
	
	Pracovní registry jsou uspořádány do struktury zásobníku.
	Přistupuje se k vrcholu tohoto zásobníku.
	Využití pro vyhodnocení výrazů a vnořená volání podprogramů.
	Většina instrukcí nemá operand (použije se implicitně např. vrchní 2 registry zásobníku).
	
	\textbf{Výhody:}
	\begin{itemize}
		\item jednoduchá a efektivní adresace operandů
		\item krátké instrukce
		\item krátké programy
		\item jednoduché dekódování instrukcí
		\item snadno lze napsat neoptimalizující překladač
	\end{itemize}
	
	\textbf{Nevýhody:}
	\begin{itemize}
		\item nelze náhodně přistupovat k lokálním datům
		\item omezený paralelismus --- zásobník je sekvenční
		\item přístupy do paměti je těžké minimalizovat
	\end{itemize}
	
	\item ISA orientovaná na registry pro všeobecné použití
	
	Dnes převládá. GPR --- General Purpose Registers. Typicky 2 nebo 3 operandy.
	
	\textbf{Výhody:}
	\begin{itemize}
		\item registry (a cache) jsou rychlejší než paměť
		\item k registrům lze přistupovat náhodně
		\item registry mohou obsahovat mezivýsledky a lokální proměnné
		\item méně častý přístup do paměti
	\end{itemize}
	
	\textbf{Nevýhody:}
	\begin{itemize}
		\item složitější překladač (optimalizace pro použití registrů)
		\item přepnutí kontextu trvá déle
	\end{itemize}
\end{itemize}

\subsubsection*{Paměťový subsystém počítače}

\begin{itemize}
	\item cache (skrytá paměť) --- rychlá, drahá, umístěna blíž k procesoru
	\item hlavní paměť --- pomalejší, levnější, větší
	\item vnější paměť --- pomalá, velká
	\item záložní paměť (CD, DVD, flash, magnetické pásky)
\end{itemize}

\begin{itemize}
	\item RAM --- random access memory (přístup adresou)
	\item CAM --- content adressable memory (přístup klíčem)
\end{itemize}

\subsubsection*{Paměťová hierarchie}

\begin{itemize}
	\item registry
	\item L1 cache (SRAM)
	\item L2 cache (SRAM)
	\item L3 cache (SRAM)
	\item hlavní paměť (DRAM)
	\item HDD, SSD
	\item Mass storage (optical disks, tapes)
	\item Remote storage (cloud)
\end{itemize}

\subsubsection*{Cache}

Kopie často používaných dat z hlavní paměti

\begin{itemize}
	\item časová lokalita 
	
	Data, ke kterým bylo právě přistupováno, budou pravděpodobně brzy potřeba znovu.
	
	\item prostorová lokalita
	
	Po přístupu k nějakým datům se pravděpodobně budou používat i vedlejší data.
\end{itemize}