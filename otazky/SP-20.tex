\subsection{SP-20 (AG1 + PA1)}
Časová a paměťová složitost algoritmů. Algoritmy vyhledávání (sekvenční, půlením intervalu), slučování a řazení (BubbleSort, SelectSort, InsertSort, MergeSort, QuickSort). Dolní mez složitosti řazení v porovnávacím modelu. Řazení v lineárním čase.

\subsubsection*{Složitost}
\begin{itemize}
	\item vyjadřuje závislost potřebného času / paměti pro vykonání algoritmu v závislosti na velikosti vstupních dat
	\item popisuje se asymptotickým horním odhadem
	\item typicky se řeší nejhorší možný průběh algoritmu
	\item asymptotický odhad nám zjednodušuje vyjádření složitosti --- zanedbávají se multiplikativní konstanty, v případě složeného výrazu se uvažuje pouze nejvýraznější člen
\end{itemize}

\subsubsection*{Vyhledávání}
\begin{itemize}
	\item sekvenční
	\begin{itemize}
		\item procházení prvků popořadě, dokud se nenajde hledaný nebo neprojdou všechny
		\item neklade žádné nároky na seřazení prvků
		\item časová složitost lineární vzhledem k datům, tedy $O(n)$
	\end{itemize}
	\item půlení intervalu
	\begin{itemize}
		\item binární vyhledávání
		\item vyžaduje seřazené prvky
		\item vždy zkontroluje prvek uprostřed aktuálního intervalu, podle porovnání s hledaným prvkem pokračuje dále do levé nebo pravé poloviny (zanoří se)
		\item časová složitost logaritmická, tedy $O(\log n)$ 
	\end{itemize}
\end{itemize}

\subsubsection*{Slučování a řazení}
\begin{itemize}
	\item bubble sort
	\begin{itemize}
		\item postupně projde všechny prvky, vždy porovná sousední 2, pokud jsou seřazené špatně tak je prohodí
		\item tyto průchody všemi prvky jsou opakovány dokud se alespoň jednou za průchod něco prohodí
		\item in-place algoritmus (nevyžaduje paměť navíc)
		\item složitost $O(n^2)$
		\item stabilní algoritmus (pořadí prvků se stejnými klíči zůstane stejné)
		\item citlivý na vstupní data (doba běhu ovlivněna pořadím dat na vstupu)
	\end{itemize}
	\item select sort
	\begin{itemize}
		\item rozděluje prvky na seřazené a neseřazené, z neseřazených vybere nejmenší a prohodí ho s prvním prvkem v neseřazené části, tím rozšíří seřazenou část o 1
		\item in-place algoritmus (nevyžaduje paměť navíc)
		\item složitost $O(n^2)$
		\item údajně nestabilní, asi je tím myšleno že závisí na implementaci vybírání prvku z neseřazené části
		\item necitlivý na vstupní data (doba běhu není ovlivněna pořadím dat na vstupu)
	\end{itemize}
	\item insert sort 
	\begin{itemize}
		\item opačný postup než u select sortu --- vezme první prvek neseřazené části, a vloží ho na správné místo v seřazené části (na počátku seřazená část $=$ první prvek)
		\item in-place algoritmus (nevyžaduje paměť navíc)
		\item složitost $O(n^2)$
		\item stabilní (řekl bych ale že také závisí na implementaci)
		\item citlivý na vstupní data
	\end{itemize}
	\item merge sort 
	\begin{itemize}
		\item rekurzivní algoritmus --- rozdělí prvky napůl, seřadí poloviny svým rekurzivním zavoláním, následně seřazené poloviny sloučí
		\item out-of-place algoritmus (vyžaduje paměť navíc kvůli slučování do nového pole, s rozumnou implementací stačí $O(n)$ paměti navíc)
		\item složitost $O(n \cdot \log n)$ (ale celkem vysoká multiplikativní konstanta)
		\item necitlivý na vstupní data
	\end{itemize}
	\item quick sort 
	\begin{itemize}
		\item vybere se prvek zvaný pivot (snaha je, aby byl co nejblíže mediánu všech prvků, závisí na tom efektivita celého algoritmu)
		\item nalevo od pivota se dají prvky menší jak on, napravo větší
		\item na levou i pravou část se rekurzivně použije znovu stejný proces
		\item takto se postupuje, dokud se nedojde na velikost samostatných prvků
		\item složitost $O(n \cdot \log n)$, teoreticky v nejhorším případě $O(n^2)$ (citlivý na vstupní data)
	\end{itemize}
\end{itemize}

\subsubsection*{Dolní mez složitosti řazení}
\begin{itemize}
	\item uvažujeme řazení v porovnávacím modelu
	\begin{itemize}
		\item prvky se mezi sebou mohou pouze porovnávat a přesouvat, není např. žádné omezení na rozsah hodnot
		\item prohození 2 prvků i porovnání je $O(1)$
	\end{itemize}
	\item požadujeme deterministický algoritmus, tedy musí postupovat na základě dat a předchozích akcí, ne náhodně
	\item při takových požadavcích je řazení ekvivalentní rozpoznání, o kterou permutaci z $n!$ možných jde --- hloubka nejméně $\log (n!)$,  DML magií převedeno na $\Omega (n\log n)$
	
	\item pro dolní mez vyhledávání podobný postup --- (ternární) strom možností s hloubkou max $\log_3 n$
\end{itemize}

\subsubsection*{Řazení v lineárním čase}
\begin{itemize}
	\item je možné dosáhnout, pokud nepracujeme v klasickém porovnávacím modelu
	\item algoritmus counting sort
	\begin{itemize}
		\item dosahuje lineární složitosti díky předpokladu omezené vstupní množiny hodnot
		\item řadí $n$ celých čísel z množiny např. $\{1,...,r\}$ 
		\item pro každé číslo se spočte histogram (kolikrát je ve vstupu obsaženo)
		\item z histogramu se spočítají pozice prvků ve výsledném poli, a v druhém průchodu se přesunou na správná místa
		\item složitost $\Theta (n + r)$ (jak paměťová, tak časová)
		\item necitlivý (datově) a není in-place
		\item stabilní
	\end{itemize}
\end{itemize}
