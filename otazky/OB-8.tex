\subsection{OB-8 (BEK)}
Zranitelnosti desktopových aplikací: Přetečení bufferu, DLL hijacking, chyby v C.

\subsubsection*{Buffer overflow}
\begin{itemize}
    \item přetečení bufferu nastává, přepíšeme-li jeden či více bytů za koncem bufferu
    \item přetečení může být jak na zásobníku, tak na haldě
    \item tím se mohou pozměnit důležitá data, např. návratová adresa, zálogy registrů...
    \item obrany:
    \begin{itemize}
        \item DEP (řeší OS) --- Data Execution Prevention --- stránky zásobníku jsou nespustitelné
        \item ASLR (řeší OS) --- Address Space Layout Randomization --- náhodné rozvržení adresního prostoru, takže adres haldy, zásobníku i instrukcí, při každém spuštění se mění
        \item kanárci (řeší si program sám) --- náhodné hodnoty hlídající zásobník, pokud se přepíšou, program zjistí že má problém
    \end{itemize}
\end{itemize}

\subsubsection*{DLL Hijacking}
\begin{itemize}
    \item jedná se o podvržení dynamické knihovny
    \item útočník vytvoří dynamickou knihovnu se stejným názvem jako knihovna, kterou hledá software, a dle algoritmu vyhledávání dll se může stát, že útočníkova verze bude upřednostněna
    \item existuje indirect varianta, kdy útočník cílí na použití své dll jinou dll, která je aplikací vyžadována
    \item bránit se lze nastavením konkrétního místa, kde se DLL má hledat (a nestahovat blbosti)
\end{itemize}

\subsubsection*{Chyby v C}
\begin{itemize}
    \item v c existuje mnoho různých funkcí pro práci se "stringy" (pole znaků)
    \item základní varianty těchto funkcí vůbec neřeší délku bufferu
    \item existují bezpečnější varianty které řeší velikost bufferu --- i ty ale musí být použity správně
    \item pozor na ukončovací nulu, na různých platformách se k ní přistupuje různě
    \item knihovna StrSafe --- Windows knihovna pro bezpečnou práci se stringy, systematické a jasné pojmenování
    \item pozor na přetečení datového typu
    \item pozor na vstup, vstup je zlo
    \item jak mít bezpečnější kód (v C)?
    \begin{itemize}
        \item nepoužívat některé funkce
        \item důsledně používat ochranu zásobníku
        \item na každou skupinu funkcí/třídu napsat testy
        \item nechávat kód kontrolovat další osobou
        \item statická a dynamická analýza kódu
        
    \end{itemize}
\end{itemize}