\subsection{OB-16 (EHA)}
Základní typy zranitelností webových aplikací, jejich testování a náprava. Bezpečnostní mechanismy webových prohlížečů.

\subsubsection*{Webové aplikace}
\begin{itemize}
    \item struktura: klient (browser), server (webový server)
    \item používané protokoly: DNS, TCP, TLS, DTLS, HTTP, WebSocket, ...
    \item dále mnoho různých technologií, komponent jak na serveru tak na klientu --- mnoho míst pro zranitelnosti
\end{itemize}

\subsubsection*{Základní zranitelnosti}
\begin{itemize}
    \item Cross Site Request Forgery (CSRF/XSRF)
    \begin{itemize}
        \item Útočník ze své stránky (tedy z klienta oběti) pošle request na jinou stránku/službu, kde pokud je oběť přihlášena, může tento request uspět a tím se provede nějaká akce, např se pošlou útočníkovi peníze
        \item možnosti testování moc nejsou, obrana je na prohlížeči (můžem otestovat max ten)
        \item prohlížeč se může (musí) bránit tak, že nedovolí posílání requestů na jinou stránku/zdroj (cross origin)
    \end{itemize}
    \item Cross Site Scripting (XSS)
    \begin{itemize}
        \item stored XSS
        \begin{itemize}
            \item v databázi je uložený kus javascript kódu, který se spustí při načtení a zobrazení klientovi (např. jméno uživatele, v nějakém příspěvku...)
            \item možnosti řešení: odebrat problematické sekvence znaků (může narušovat funkčnost, špatné ře\-še\-ní), vyescapovat nebezpečné znaky
            \item jak testovat? prostě to zkusim několika způsoby vložit kde to jde (kde se něco ukládá a zobrazuje)
        \end{itemize}
        \item reflected XSS
        \begin{itemize}
            \item javascript není uložený, ale dostane se do nějaké zobrazované proměnné, která může být ov\-lád\-nu\-ta, jako např. hodnota z URL adresy
            \item jak řešit? z klientského pohledu --- pozor na to co může přijít zvenčí za vstup --- jako správce je lepší takové proměnné navázat jinak, např. do payloadu POST requestu, a/nebo kontrolovat jejich hodnotu (VSTUP JE ZLO)
            \item jak testovat? najít možná zranitelná místa (vstup z url) a vyzkoušet
        \end{itemize}
        \item DOM-based XSS
        \begin{itemize}
            \item založené na dynamicky generovaném obsahu, tedy jiný "typ" proměnné ovlivňující vzhled/obsah stránky, která může být ovládnuta uživatelem/útočníkem
            \item podobně jako výše --- kontrola vstupů pro obranu, identifikace slabých míst a vyzkoušení pro test
        \end{itemize}
    \end{itemize}
    \item SQL Injection
    \begin{itemize}
        \item vložení SQL kódu do nějakého vstupu, výsledkem je ovlivnění výsledku / nechtěné změny v DB / vyzrazení informací
        \item jak se bránit? ošetřovat vstupy! escapování kritických znaků, prepared statements (nejlepší)
        \item jak testovat? najdu vstup co možná jde do databáze, vyzkouším
    \end{itemize}
    \item Command Injection
    \begin{itemize}
        \item uživatel ovlivňuje nějakým vstupem příkaz na serveru aplikace, může provádět nechtěné příkazy
        \item jak se bránit? blacklist/whitelist znaků, specifická kontrola vstupu pro konkrétní data ... případně upustit od této funkcionality
        \item jak hledat? podobně jako předchozí...
    \end{itemize}
    \item File uploads
    \begin{itemize}
        \item ovlivňování jména nahrávaného souboru, možné nahrát někam kam autor nechtěl, něco přepsat...
        \item obrana: kontrola vstupu, blacklist/whitelist znaků
        \item jak testovat? pokud se něco tváří jako jméno souboru, vyzkoušet různé varianty vyšších složek a podobně
    \end{itemize}
    \item File inclusion
    \begin{itemize}
        \item uživatel může do aplikace přidat kód, může to být způsobeno nebezpečnými přímými odkazy
        \item může přidat serverový kód, klientský kód, nebo např. /dev/random (způsobit DoS) nebo /etc/passwd (information disclosure)
        \item jak bránit? nenechat uživatele volit soubory přímo, ale přes nepřímý odkaz, např. ID
    \end{itemize}
    \item Server side request forgery (SSRF)
    \begin{itemize}
        \item útočník může donutit aplikaci k vykonávání požadavků na další místa, např. v lokální síti serveru
        \item lze díky tomu např. scanovat vnitřní síť nebo získat citlivá data
        \item řešení: nepřímé odkazy, validace vstupů
    \end{itemize}
    \item XML external entity (XXE)
    \begin{itemize}
        \item při parsování XML souboru lze přidat odkaz na externí XML entitu, lze tím vykonat kód, přidat soubor...
        \item obrana: správně nakonfigurovat parser, nebo zakázat externí entity úplně
    \end{itemize}
    \item server-side template injection (SSTI)
    \begin{itemize}
        \item do šablony (template) se dá zadat něco, co způsobí vykonávání kódu na serveru
        \item obrana: použití logic-less template engine
    \end{itemize}
    \item open redirection
    \begin{itemize}
        \item neověřený uživatel může změnit, kam stránka jiného uživatele po přihlášení přesměruje
        \item obrana: vyhnout se možnosti uživatelského nastavení přesměrování
    \end{itemize}
    \item formula injection
    \begin{itemize}
        \item týká se aplikací exportujících tabulky, .csv nebo .xlsx --- pokud se vyexportují neověřená data od útočníka, můžou se pak na PC oběti spustit nebezpečná makra
        \item prostě ověřovat vstup lol
    \end{itemize}
\end{itemize}

\subsubsection*{Obrany prohlížeče}
\begin{itemize}
    \item security headers
    \item Strict-Transport-Security --- řeší povinnost HTTPS
    \item X-Frame-Options --- povolení vykreslování částí stránek v rámečcích (jako iframe)
    \item X-Content-Type-Options --- definování typu obsahu, browser podle toho k obsahu přistupuje
    \item Content-Security-Policy --- kontroluje povolený obsah k načtení
\end{itemize}