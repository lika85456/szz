\subsection{OB-3 (ADU)}
Procesy a systémové služby v unixových operačních systémech: hierarchie a vzájemné vazby, limity, zapínání a vypínání systému, logování aktivit systému.

\textbf{Procesy:}
\begin{itemize}
	\item vykonávané programy
	\item mají svoje ID (PID), id uživatele (UID), id rodiče (PPID) ...
	\item každý proces někdo vytvořil, tedy každý proces má nějaké PPID --- init proces je první proces spuštěn při bootu, má PID = 1 a PPID = 0
	\item rodič typicky čeká, až mu dítě předá návratový kód
	\item daemon = systémový proces (běží na pozadí bez nutnosti vstupu ... jako u windows služby)
	\item orphan = proces, jehož rodič už neexistuje --- adoptuje se pod proces init
	\item zombie = proces, který již skončil, ale ještě si rodič nepřevzal návratovou hodnotu
	\item procesy si navzájem mohou posílat signály
\end{itemize}

\textbf{Limity:}
\begin{itemize}
	\item soft limit --- limit mezi 0 a hard limitem, lze uživatelsky přenastavit
	\item hard limit --- může změnit jen root, nelze překročit
	\item lze nastavit max počet procesů pro uživatele a další omezení využití zdrojů (paměť, velikost souborů, počet file descriptorů...)
	\item příkaz ulimit
\end{itemize}

\textbf{BOOT}
\begin{itemize}
	\item Firmware fáze
	\begin{itemize}
		\item POST (Power On Self Test)
		\item inicializace HW a driverů
		\item výběr bootovacího zařízení
		\item načtení a spuštění bootovacího kódu
	\end{itemize}
	\item Boot-loader fáze
	\begin{itemize}
		\item nalezení a načtení boot programu
		\item boot program se nahrává do pevně daných adres v paměti
		\item kontrola řízení se předá programu
	\end{itemize}
	\item GRUB (Grand Unified Boot Loader) fáze
	\begin{itemize}
		\item načte se GRUB konfigurace
		\item vybere se odkud/co bootovat
		\item kernel se načte a spustí
	\end{itemize}
	\item Kernel fáze
	\begin{itemize}
		\item inicializace datových struktur jádra
		\item načtení driverů, inicializace HW
		\item mount kořenového filesystému
		\item načtení konfigurace jádra
		\item načtení modulů
		\item vytvoření init procesu
	\end{itemize}
	
	\newpage
	\item Init fáze
	\begin{itemize}
		\item dříve spouštění start/stop scriptů dle úrovní běhu systému (run levels / milestones)
		\begin{itemize}
			\item 0 --- systém vypnutý
			\item 1, s, S --- single user mode
			\item 2 --- multi user mode
			\item 3 --- multi user mode + síť
			\item 4 --- nepoužívaný (občas GUI)
			\item 5 --- vypnutí
			\item 6 --- restart
		\end{itemize}
		\item v dnešní době spouštění služeb
		\item konfigurace procesu init --- /etc/inittab
		\item init se přes fork() a exec() naklonuje a spouští další procesy
	\end{itemize}
\end{itemize}

\textbf{Vypínání systému:}
\begin{itemize}
    \item shutdown \textit{level timeout -y} (přívětivý jak k uživatelům a aplikacím, tak k systému)
    \item init \textit{level} (přívětivý k aplikacím a systému)
    \item poweroff, restart (přívětivé k systému)
    \item vypnout napájení (fuj)
\end{itemize}

\textbf{Logování:}
\begin{itemize}
	\item servrová logovací služba s dlouhou historií
	\item umožňuje oddělit SW generující zprávy, SW který je ukládá a SW který je analyzuje a nahlašuje
	\item konfigurace: /etc/syslog.conf
	\item logovací složky: /var/log a /var/adm
	\item mnoho variant
	\item záznamy v logu mají info o službě a závažnosti logu (alert, critical, error, warning, notice, info, debug)
\end{itemize}