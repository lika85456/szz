\subsection{OB-24 (ZSB)}
Forenzní analýza souborových systémů, principy, možnosti obnovy smazaných dat.

\subsubsection*{Soubory}
\begin{itemize}
	\item pojmenovaná množina dat přesně definovaného formátu, uložená na datovém médiu
	\item na uživatelské úrovni lze typ souboru rozeznat dle přípony
	\item strojově lze typ rozeznat dle přípony a podle hlavičky uvnitř souboru
	\item existují otevřeně specifikované formáty a uzavřené formáty
	\item typy souborů:
	\begin{itemize}
		\item textové/dokumentové soubory (txt, doc, docx, pdf, xls, pages, html,...)
		\item obrázkové soubory (rastr --- jpg, png, bmb,...) (vektor --- psd, cdr, ai,...)
		\item zvukové soubory (wav, mp3, midi (neni zvukovej lol), wma,...)
		\item video (avi, mpg, mov, mp4,...)
		\item archivy (zip, rar, gz,...)
		\item spustitelné soubory (exe, dmg,...)
		\item speciální...
	\end{itemize}
\end{itemize}

\subsubsection*{Souborové systémy}
\begin{itemize}
	\item organizuje data (soubory do složek)
	\item obsahuje metadata popisující strukturu a některé vlastnosti souborů
	\item příklady: FAT32, NTFS, EXT3, EXFAT...
\end{itemize}

\subsubsection*{Obnova smazaných dat}
\begin{itemize}
	\item co jsou to smazaná data?
	\begin{itemize}
		\item nechtěné smazání souborů či složek
		\item reformátování filesystému
		\item poškození superbloku/FAT/MFT
		\item poškození záznamů o oddílech (partitions)
		\item fyzické poškození hard disku
	\end{itemize}
	\item informace na discích se nemažou, když se smaže soubor
	\item segmenty/clustery se pouze označí jako volné
	\item obnovit se dá 2 způsoby --- pomocí metadat filesystému, a pomocí hledání hlaviček známých typů souborů
	\item obnovení pomoví metadat vydá strukturu složek, názvy složek a souborů, metadata souborů a složek
	\item obnovení pomocí file carvingu (hledání pomocí hlaviček) vydá surové bloky dat bez metadat a názvů složek a souborů
\end{itemize}