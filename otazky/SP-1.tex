\subsection{SP-1 (AAG)}
Regulární jazyky: Deterministické a nedeterministické konečné automaty. Determinizace konečného automatu. Minimalizace deterministického konečného automatu. Operace s konečnými automaty. Regulární gramatiky, regulární výrazy, regulární rovnice.

\subsubsection*{Regulární jazyky}
\begin{itemize}
	\item jsou podmnožinou bezkontextových, kontextových i rekurzivně spočetných jazyků
	\item jazyk je regulární právě tehdy, kdy lze generovat regulární gramatikou
	\item regulární jazyk lze přijmout (ne)deterministickým konečným automatem
	\item regulární jazyk lze popsat regulárním výrazem
	\item regulární jazyky jsou uzavřené pro všechny operace: sjednocení, průnik, doplněk, rozdíl, zřetězení, iterace
	\item konečný jazyk je vždy regulární
\end{itemize}

\subsubsection*{Gramatiky}
\begin{itemize}
	\item prostředek pro popis jazyků
	\item uspořádaná čtveřice: $G = (N, \Sigma, P, S)$, kde:
	\begin{itemize}
		\item $N$ je konečná neprázdná množina neterminálních symbolů
		\item $\Sigma$ je abeceda symbolů, které nazýváme terminální symboly ---abeceda generovaného jazyka
		\item $N \cap \Sigma = \emptyset$
		\item $P$ je konečná množina přepisovacích pravidel ve tvaru $\alpha A \beta \rightarrow \gamma \quad (\alpha, \beta	, \gamma \in (N \cup \Sigma)^*, A \in N)$
		\item $S$ je počáteční symbol gramatiky
	\end{itemize}
	\item regulární gramatika má pravidla ve tvaru $A \rightarrow a$ nebo $A \rightarrow aB \quad (a \in \Sigma, A,B \in N)$
	\item jediná povolená výjimka je pravidlo $S \rightarrow \varepsilon$, které může být přítomno pouze pokud symbol $S$ není na pravé straně žádného pravidla
	\item regulární gramatika je tedy nezkracující
\end{itemize}

\subsubsection*{Konečné automaty}
\begin{itemize}
	\item výpočetní model skládající se z řídící jednotky (stavy a přechodová funkce), read-only vstupní pásky a čtecí hlavy, která čte jednotlivé symboly na pásce
	\item na počátku je automat v počátečním stavu, a hlava ukazuje na první symbol na pásce
	\item v každém kroku se přečte symbol, na základě něj se může změnit stav automatu, a hlava se přesune na další znak
	\item tento proces se opakuje, dokud existuje na vstupu nějaký symbol
	\item formálně: konečný automat je uspořádaná pětice: $M = (Q, \Sigma, \delta, q_0, F)$, kde:
	\begin{itemize}
		\item $Q$ je konečná neprázdná množina stavů
		\item $\Sigma$ je konečná vstupní abeceda
		\item $\delta$ je přechodová funkce
		\begin{itemize}
			\item pro deterministický KA je to zobrazení z množiny $Q \times \Sigma$ do množiny stavů $Q$ (tedy $\delta: Q \times \Sigma \rightarrow Q$)
			\item pro nedeterministický KA je to zobrazení z $Q \times \Sigma$ do všech podmnožin $Q$
			\item NKA se dá navíc rozšířit o $\varepsilon$-přechody, přechodová funkce tedy je z množiny $Q \times (\Sigma \cup \varepsilon)$
			\item NKA se také dá rozšířit tak, že má více počátečních stavů
		\end{itemize}
		\item $q_0$ je počáteční stav
		\item $F \subset Q$ je množina koncových stavů
	\end{itemize}
	\item NKA i s $\varepsilon$-přechody a více počátečními stavy lze vždy převést na jednodušší variantu, tedy až na DKA
	\item KA lze zapsat více různými způsoby, liší se v zápisu přechodové funkce (formálně, orientovaný ohodnocený graf, tabulka,...)
	\item homogenní KA --- do konkrétního stavu se lze vždy dostat za pomocí právě jednoho symbolu
	\item výsledkem "práce" automatu je přijmutí/nepřijmutí řetězce (rozhodnutí, zda patří do jazyka)
	\item řetězec se přijme, pokud se přečte celý, a automat skončí v některém z koncových stavů
\end{itemize}

\subsubsection*{Úpravy konečných automatů}
\begin{itemize}
	\item nedosažitelný stav: stav, do kterého se nedá dostat z počátečního stavu žádnou posloupností přechodů
	\item zbytečný stav: stav, ze kterého se nedá dostat do žádného koncového stavu
	\item nedosažitelné a zbytečné stavy lze odstranit, a nemá to vliv na jazyk přijímaný automatem
	\item odstranění $\varepsilon$-přechodů
	\begin{itemize}
		\item vypočteme funkci $\varepsilon$-closure pro všechny stavy (pro daný stav vrací množinu stavů, kam se lze dostat bez přečtení znaku ze vstupu, tedy kam se dá dostat $\varepsilon$-přechody, obsahuje i sebe sama)
		\item z automatu odstraníme $\varepsilon$- přechody, a pomocí $\varepsilon$-closure upravíme přechodovou funkci automatu (pokud pro daný stav $S$ je $\varepsilon$-closure $\{S, A\}$, přidáme přechody ze stavu $S$ do stavů, kam vedou přechody z $A$)
		\item upravíme množinu koncových stavů --- pokud stav v $\varepsilon$-closure měl původní koncový stav, je teď nově také koncový
	\end{itemize}
	\item odstranění více počátečních stavů --- přidáme nový, jediný počáteční, a z něj uděláme $\varepsilon$-přechody do původních počátečních
	\item determinizace (převod NKA na DKA)
	\begin{itemize}
		\item na vstupu předpokládáme NDA s jedním počátečním stavem a bez $\varepsilon$-přechodů
		\item tvoříme nová (D)KA tak, že procházíme původní NKA po množinách jeho stavů (stav nového DKA --- množina všech původních stavů, kde NKA v tu chvíli může být)
		\item pro každou takto objevenou množinu dále tvoříme přechodovou funkci
		\item ve chvíli, kdy již máme všechny objevené množiny "zpracované", máme hotovo
	\end{itemize}
	\item minimalizace
	\begin{itemize}
		\item ekvivalentní stavy --- stavy, po jejichž sloučení se nezmění přijímaný jazyk
		\item minimální DKA --- takový automat pro daný jakzyk $L$, který má co nejméně stavů, tedy nemá ani nedosažitelné, zbytečné ani ekvivalentní stavy
		\item nejprve odstraníme nedosažitelné a zbytečné stavy (zjistíme jednoduše BFS z počátečního/koncových stavů)
		\item rozdělíme stavy na koncové a nekoncové, stavy přejmenujeme podle toho, do které skupiny patří
		\item pro tyto přejmenované stavy vyplníme přechodovou funkci podle tohoto nového pojmenování
		\item projdeme, zda všechny stavy v rámci jedné skupiny mají shodné přechody
		\item pokud najdeme ve skupině odlišný stav, vytvoříme pro něj novou skupinu, případně pokud už jsme v aktuálním průchodu narazili na "stejně odlišný" stav, tak je dáme do stejné nové skupiny
		\item předchozí 3 kroky opakujeme, dokud se nějakým způsobem mění složení skupin a pojmenování stavů
	\end{itemize}
	\item ekvivalentní KA --- převedeme na minimální DKA, a nalezneme isomorfismus
\end{itemize}

\subsubsection*{Skládání KA}
\begin{itemize}
	\item sjednocení
	\begin{itemize}
		\item pokud jsou vstupem NKA, NKA s $\varepsilon$-přechody nebo DKA --- buď prostě více počátečních stavů, nebo nový 1 počáteční stav a $\varepsilon$-přechody z něj
		\item pokud vstupem jsou úplně určené DKA nebo NKA, lze využít paralelní činnosti
		\begin{itemize}
			\item podobné jako determinizace
			\item začne se ve stavu, který představuje počáteční stavy obou vstupních automatů
			\item postupuje se podle přechodových funkcí vstupních automatů, každý nový stav je kombinací stavů z obou původních automatů
		\end{itemize}
	\end{itemize}
	\item průnik
	\begin{itemize}
		\item pomocí paralelní činnosti, podobně jako sjednocení, ale automaty nemusí být úplně určené
		\item tam kde v jednom automatu není přechod, nebude existovat ani ve výsledném automatu
		\item koncové stavy jsou navíc jen ty, které "obsahují" pouze koncové stavy obou vstupních automatů
	\end{itemize}
	\item doplněk jazyka --- vstup je úplně určený DKA, prohodíme koncové a nekoncové stavy
	\item rozdíl --- $L(M_1) \setminus L(M_2) = L(M_1) \cap \overline{L(M_2)}$
	\item součin (zřetězení) --- ze všech koncových stavů $M_1$ $\varepsilon$-přechody do počátečního stavu $M_2$, koncové stavy budou jen $M_2$ a počáteční jen $M_1$
	\item iterace --- nový počáteční stav (zároveň koncový), $\varepsilon$-přechod do původního počátečního, ze všech koncových vedeme $\varepsilon$-přechody do nového počátečního
\end{itemize}

\subsubsection*{Regulární výrazy}
\begin{itemize}
	\item $\varnothing, \varepsilon, a$ jsou regulární výrazy pro všechna $a \in \Sigma$
	\item pokud jsou $x, y$ regulární výrazy nad $\Sigma$, pak $(x + y), (x \cdot y)$ a $(x)^*$ jsou regulární výrazy nad $\Sigma$
	\item hodnota regulárního výrazu je regulární jazyk, který daný výraz reprezentuje
	\begin{itemize}
		\item $h(\varnothing) = \varnothing$ --- prázdný výraz reprezentuje prázdný jazyk
		\item $h(\varepsilon) = \{\varepsilon\}$
		\item $h(a) = \{a\}$
		\item $h(x + y) = h(x) \cup h(y)$
		\item $h(x\cdot y) = h(x)\cdot h(y)$
		\item $h(x^*) = h(x)^*$
	\end{itemize}
	\item lze podle různých pravidel zjednodušovat (např $\varnothing$ se chová jako 0 při násobení i sčítání, $\varepsilon$ se chová jako 1 při násobení)
	\item regulární rovnice
	\begin{itemize}
		\item levá regulární rovnice $x = x\alpha + \beta$
		\item pravá regulární rovnice $x = \alpha x + \beta$
		\item $\alpha, \beta$ jsou známé výrazy, $x$ neznámý
		\item potřebujeme dostat neznámý výraz "ven" z podvýrazů, aby se dala aplikovat levá/pravá rovnice
		\item soustavu řešíme tak, že co jde dosadit rovnou do dalších rovnic dosadíme, jinak jednu vyřešíme klasickým způsobem a pak dosadíme
	\end{itemize}
	\item derivace regulárních výrazů --- odebírání předpony 
	\item pokud nejde odebrat konkrétní předponu, výsledek je $\varnothing$
	\item pravidla:
	\begin{itemize}
		\item $\frac{\text{d}V}{\text{d}\varepsilon} = V$
		\item $\frac{\text{d}\varepsilon}{\text{d}a} = \varnothing$
		\item $\frac{\text{d}\varnothing}{\text{d}a} = \varnothing$
		\item $\frac{\text{d}b}{\text{d}a} = \varnothing$
		\item $\frac{\text{d}a}{\text{d}a} = \varepsilon$
		\item $\frac{\text{d}(U + V)}{\text{d}a} = \frac{\text{d}U}{\text{d}a} + \frac{\text{d}V}{\text{d}a}$
		\item $\frac{\text{d}(UV)}{\text{d}a} = \frac{\text{d}U}{\text{d}a}V + \{\frac{\text{d}V}{\text{d}a}:\varepsilon \in h(U)\}$
		\item $\frac{\text{d}(V^*)}{\text{d}a} = \frac{\text{d}V}{\text{d}a}\cdot V^*$
	\end{itemize}
\end{itemize}

\subsubsection*{Převody}
\begin{itemize}
	\item RG $\leftrightarrow$ KA --- stavy jsou většinou stejné jako neterminály, jen se musí pořešit koncové stavy
	\item RV $\rightarrow$ KA
	\begin{itemize}
		\item metoda sousedů
		\begin{itemize}
			\item každý znak se "očísluje" --- tedy každý znak se považuje za jiný
			\item vytvoří se množiny znaků co můžou být na začátku, na konci, a množina možných sousedních znaků
			\item každý znak má v automatu vlastní stav + 1 nový počáteční stav, koncové jsou koncové, z nového počátečního (pokud $\varepsilon \in L$ tak zároveň koncového) hrany do počátečních znaků, pak se doplní hrany podle množiny sousedů
		\end{itemize}
		\item metoda derivací --- postupně derivujeme všemi znaky abecedy, každý nový regulární výraz je nový stav
		\item metoda postupné konstrukce --- sestavujeme automat od elementárních výrazů
	\end{itemize}
	\item KA $\rightarrow$ RV
	\begin{itemize}
		\item metoda regulárních rovnic --- odchozí přechody --- sestavíme soustavu rovnic, každý stav je neznámá, pro odchozí přechody stavíme pravé rovnice, ke koncovým stavům přidáme $+\varepsilon$ a výsledkem je výraz pro počáteční stav
		\item RR --- příchozí přechody --- podobně jako odchozí, ale staví se levé rovnice, $+\varepsilon$ se přidá k počátečnímu stavu, a výsledkem je sjednocení (+) výrazů koncových stavů
		\item eliminace stavů --- přidáme nový počáteční stav (a také nový koncový, pokud původních koncových je víc) a postupně odebíráme stavy s tím, že postupně sestavujeme regulární výraz
	\end{itemize}
	\item RG $\rightarrow$ RV
	\begin{itemize}
		\item metoda regulárních rovnic --- vezmeme pravidla, předěláme na reg. rovnice a vyřešíme pro počáteční neterminál
		\item eliminace neterminálních symbolů --- jako eliminace stavů
	\end{itemize}
	\item RV $\rightarrow$ RG
	\begin{itemize}
		\item metoda derivací
		\item postupná konstrukce
	\end{itemize}
\end{itemize}

\subsubsection*{Pumping lemma}
\begin{itemize}
	\item regulární jazyk $\Rightarrow$ pumpující vlastnost
	\item používá se pro důkaz neregularity jazyka
	\item najdeme nějakou větu z jazyka, rozdělíme ji na 3 části $xyz$, $|xy| \leq p$, $|y| \geq 1$, a najdeme $k$ takové, aby $xy^k z \not\in L$
\end{itemize}
